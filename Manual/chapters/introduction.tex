\setchapterstyle{kao}
\setchapterpreamble[u]{\margintoc}
\chapter{Introduction}
\labch{intro}

\section{The Main Ideas}

This document is intended to provide the physical and mathematical background concerning this simulation tool kit. It will provide the necessary information to operate the tool, tailor it to specific use cases, and develop the source code to augment it.

The first chapter will give a brief introduction in the physical and mathematical background required to build a three degree of freedom or a six degree of freedom simulation respectively. Compiling this tool lessons have been learned how to address certain problems, how to implement mathematical models in software in order to keep the code flexible and performant alike. This section will partly be a best picks from the available literature, partly methods defined within and for this particular project

\section{Background}

\section{What This Toolkit Does}
\labsec{does}

The \Class{kaobook} class focuses more about the document structure than 
about the style. Indeed, it is a well-known \LaTeX\xspace principle that 
structure and style should be separated as much as possible (see also 
\vrefsec{doesnot}). This means that this class will only provide 
commands, environments and in general, the opportunity to do things, 
which the user may or may not use. Actually, some stylistic matters are 
embedded in the class, but the user is able to customise them with ease.

The main features are the following:

\begin{description}
	\item[Page Layout] The text width is reduced to improve readability 
	and make space for the margins, where any sort of elements can be 
	displayed.
	\item[Chapter Headings] As opposed to Tufte-Latex, we provide a 
	variety of chapter headings among which to choose; examples will be 
	seen in later chapters.
	\item[Page Headers] They span the whole page, margins included, and, 
	in twoside mode, display alternatively the chapter and the section 
	name.\sidenote[-2mm][]{This is another departure from Tufte's 
	design.}
	\item[Matters] The commands \Command{frontmatter}, 
	\Command{mainmatter} and \Command{backmatter} have been redefined in 
	order to have automatically wide margins in the main matter, and 
	narrow margins in the front and back matters. However, the page 
	style can be changed at any moment, even in the middle of the 
	document.
	\item[Margin text] We provide commands \Command{sidenote} and 
	\Command{marginnote} to put text in the 
	margins.\sidenote[-2mm][]{Sidenotes (like this!) are numbered while 
	marginnotes are not}
	\item[Margin figs/tabs] A couple of useful environments is 
	\Environment{marginfigure} and \Environment{margintable}, which, not 
	surprisingly, allow you to put figures and tables in the margins 
	(\cfr \reffig{marginmonalisa}).
	\item[Margin toc] Finally, since we have wide margins, why don't add 
	a little table of contents in them? See \Command{margintoc} for 
	that.
	\item[Hyperref] \Package{hyperref} is loaded and by default we try 
	to add bookmarks in a sensible way; in particular, the bookmarks 
	levels are automatically reset at \Command{appendix} and 
	\Command{backmatter}. Moreover, we also provide a small package to 
	ease the hyperreferencing of other parts of the text.
	\item[Bibliography] We want the reader to be able to know what has 
	been cited without having to go to the end of the document every 
	time, so citations go in the margins as well as at the end, as in 
	Tufte-Latex. Unlike that class, however, you are free to customise 
	the citations as you wish.
\end{description}

\begin{marginfigure}[-5.5cm]
	\includegraphics{monalisa}
	\caption[The Mona Lisa]{The Mona Lisa.\\ 
	\url{https://commons.wikimedia.org/wiki/File:Mona_Lisa,_by_Leonardo_da_Vinci,_from_C2RMF_retouched.jpg}}
	\labfig{marginmonalisa}
\end{marginfigure}

The order of the title pages, table of contents and preface can be 
easily changed, as in aly \LaTeX\ document. In addition, the class is 
based on \KOMAScript's \Class{scrbook}, therefore it inherits all the 
goodies of that.


